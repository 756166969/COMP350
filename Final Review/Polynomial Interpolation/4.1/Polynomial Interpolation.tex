\documentclass [9 pt]{article}
\usepackage[margin = 1in]{geometry}
\usepackage{amsfonts}
\usepackage{amsthm}
\usepackage{bbm}
\usepackage{amsmath}
\usepackage{arydshln}
\usepackage[utf8]{inputenc}
\usepackage{graphicx}
\usepackage{enumerate}
\usepackage{color}
\usepackage[dvipsnames]{xcolor}
\usepackage{graphicx}
\graphicspath{ {./images/} }
\usepackage{tikz}
\usepackage{xcolor}
\usepackage{listings}
\usepackage{color}
\usepackage{algorithm}
\usepackage{algpseudocode}
\usepackage{tikz}
\usepackage{booktabs}
\usepackage[framemethod = tikz]{mdframed}

\definecolor{dkgreen}{rgb}{0,0.6,0}
\definecolor{gray}{rgb}{0.5,0.5,0.5}
\definecolor{mauve}{rgb}{0.58,0,0.82}


\lstset{
  language=Matlab,
  basicstyle=\ttfamily,               
  numbers=left,                  
  stepnumber=1,                  
  numbersep=5pt,                  
  backgroundcolor=\color{white}, 
  showspaces=false,              
  showstringspaces=false,        
  showtabs=false,                
  tabsize=2,                     
  captionpos=b,                  
  breaklines=true,               
  breakatwhitespace=true,         
  title=\lstname,
  numberstyle=\tiny\color{Black},
  keywordstyle=\color{BrickRed},
  commentstyle=\color{dkgreen},
  stringstyle=\color{mauve},  
}

\theoremstyle{definition}
\newtheorem{problem}{Problem}
\newtheorem{theorem}{Theorem}
\newtheorem*{corollary}{Corollary}
\newtheorem{proposition}[theorem]{Proposition}
\newtheorem{lemma}[theorem]{Lemma}
\newtheorem{conjecture}[theorem]{Conjecture}

\newtheorem{definition}[theorem]{Definition}
\newtheorem{remark}[theorem]{Remark}
\newtheorem{example}[theorem]{Example}


\usepackage{fancyhdr}
\pagestyle{fancy}
\lhead{Yuhao Wu \quad 260711365} 
\rhead{\bfseries COMP 350 Final Review 4.1}
\cfoot{\thepage}
\renewcommand{\headrulewidth}{0.4pt}
\renewcommand{\footrulewidth}{0.4pt}


\setlength{\parindent}{0pt}


\begin{document}

\title{Polynomial Interpolation (25 points)}
\date{2018-11-30}
\author{Name: Yuhao Wu\\
ID Number: 260711365
}
\maketitle

\section*{The Vandermonde Approach}
Given $(n+1)$ points, $(x_0, y_0), (x_1, y_1), \ldots (x_n, y_n) $, there is a \textbf{ unique } polynomial $p$ of degree $\leq n$ such that $P(x_i) = y_i$
\begin{mdframed}
\begin{proof}

Let $p(x) = c_0 + c_1x + c_2 x^2 +  \ldots + c_n x^n$, then we have $A c = y$
$$\begin{bmatrix}
	1 & x_0 & x_0^2 &\ldots &x_0^n \\
	1 & x_1 & x_1^2 &\ldots &x_1^n \\
	\vdots&\vdots &\vdots &\vdots &\vdots\\
	1 & x_n & x_n^2 &\ldots &x_n^n \\
\end{bmatrix} \cdot \begin{bmatrix}
	c_0\\
	c_1\\
	\vdots\\
	c_n
\end{bmatrix} = \begin{bmatrix}
	y_0\\
	y_1\\
	\vdots\\
	y_n
\end{bmatrix}$$
As $A$ is called the \textit{ Vandermonde Matrix} and
$$ \det(A) = \prod_{ 0 \leq i\leq j \leq n}(x_j - x_i) \neq 0 $$		
Thus $A$ is non-singular and $Ac = y$ has a unique solution $c = A^{-1}y$
\end{proof}
\end{mdframed}
\textbf{Algorithm:}
\begin{itemize}
	\item[1] Form the linear system $Ac = y$ 
	\item[2] Solve $Ac = y$ by GEPP
\end{itemize}
\textbf{Cost Analysis:}
\begin{itemize}
	\item[1] For First Step, for each line of the matrix, we need $n - 1$ multiplication based on $x_0, x_1, \ldots, x_n$, thus, we need total $(n - 1) \times (n+1) \approx n^2$ flops
	\item[2] For GEPP, we need $\dfrac{2}{3}n^3$ flops, due to $A$ has some special structures, we can cost as low as $O(n \cdot \log^2(n))$
\end{itemize}
\textbf{Evaluating $p(x)$ (2n flops):}
\begin{align*}
	p(x) 
	&= c_0 + c_1\cdot x + c_2 \cdot x^2 + \ldots + c_n \cdot x^n\\
	&= c_0 + x\bigg(c_1 + x \Big( c_2 + \ldots + x(c_{n-1} + x \cdot c_n) \Big) \bigg)
\end{align*}
$p \gets c_n $\\
for $i = n - 1 : -1 : 0$\\
$\quad p \gets c_i + x\cdot p$\\
end \\







\newpage
\section*{The Lagrange Approach:}
The Lagrange form of the interpolating polynomial:
$$p(x) = \sum_{i = 0}^n \ell_i(x) \cdot y_i$$
where $\ell_i(x)$ is the cardinal polynomial defined as:
$$\ell_i(x) = \dfrac{(x - x_0) \cdot (x - x_1) \ldots (x - x_{i - 1}) \cdot (x - x_{i+1}) \ldots (x- x_n) }{(x_i - x_0) \cdot (x_i - x_1) \ldots (x_i - x_{i - 1}) \cdot (x_i - x_{i+1}) \ldots (x_i - x_n)} \quad 
\ell_i(x_j) = \begin{cases}
	0, \quad i \neq j\\
	1, \quad i = j
\end{cases}$$
We can rewrite $p(x)$:
$$ p(x) = \sum_{i = 0}^n \ell_i(x) \cdot y_i = \sum_{i = 0}^n \dfrac{y_i}{ \prod_{j = 0, j \neq i}^{n} (x_i - x_j)} \cdot \dfrac{ \prod_{j = 0}^n (x - x_j) }{x - x_i} = q(x) \cdot \sum_{i = 0}^{n} \dfrac{c_i}{ x - x_i } $$

$$where \ q(x) = \prod_{j = 0}^n (x - x_j) \quad\quad c_i =  \dfrac{y_i}{ \prod_{j = 0, j \neq i}^{n} (x_i - x_j)} $$

\textbf{Cost of Finding $c_0, c_1, \ldots, c_n$}\\
For each $i$, computing $c_i$ needs 1 division, $n$ subtraction, $n - 1$ multiplication, a total $2n$ flops.\\
So, computing all $c_i$ needs $2n \times (n + 1) \approx 2n^2$ flops.\\
\\
\textbf{Cost of Evaluating $p(x)$:} \\
Computing $q(x)$ needs $(2n + 1)$ flops ( n+1 subtraction, n multiplication )\\
Computing each $\dfrac{c_i}{x - x_i}$ needs 2 flops for each $i$, total $2\times(n+1)$ \\
Adding them together, need $n$ flops. \\
Thus, we need total $(2n + 1) + 2\times(n+1) + n \approx 5n $



\section*{The Newton Approach}

\textbf{Idea:} 
Suppose a polynomial $p_k(x)$ of degree at most $k$ has been found to interpolate $(x_0, y_0), (x_1, y_1), \ldots , (x_k, y_k)$.\\
Then, what we want to do is to find $p_{k+1}(x)$ of degree at most $k + 1$ to interpolate  $(x_0, y_0), (x_1, y_1), \ldots , (x_k, y_k), (x_{k+1}, y_{k+1})$\\
\\
Set $p_{k+1} = p_k(x) + a_{k+1} (x- x_0) \cdot(x - x_1) \cdot \ldots \cdot (x- x_k) $, where $a_{k+1}$ is to be determined.\\
\\
Obviously, we have 
$$ p_{k+1}(x_i) = p_{k}(x_i) = y_i \quad 0 \leq i \leq k $$
Setting $p_{k+1}(x_{k+1}) = y_{k+1} $, we obtain:
$$a_{k+1} = \dfrac{y_{k+1} - p_k(x_{k+1})}{ (x- x_0) \cdot(x - x_1) \cdot \ldots \cdot (x- x_k) }$$
\subsection*{The Newton form of the interpolating polynomial:}
$$p_n(x) = a_0 + a_1 \cdot (x - x_0) + a_2\cdot (x  - x_0)(x - x_1) + \ldots +a_n \cdot (x - x_0) \cdot (x - x_1) \ldots (x - x_{n - 1}) $$

\newpage
\textbf{Evaluating: (3n) flops}
\begin{align*}
	p_n(x) 
	&= a_0 + a_1 \cdot (x - x_0) + a_2\cdot (x  - x_0)(x - x_1) + \ldots +a_n \cdot (x - x_0) \cdot (x - x_1) \ldots (x - x_{n - 1}) \\
	&= a_0 +  (x - x_0) \cdot \bigg( a_1 + ( x - x_1) \Big( a_2 +  \ldots + ( a_{n - 1} + (x - x_{n - 1}) a_n) \Big) \bigg)
\end{align*}

Procedure for Evaluating $p_n(x)$ for some $x$:\\
$p \gets a_n$\\
for $i = n - 1 : -1 : 0$\\
$p \gets a_i + (x - x_i) \cdot p$\\
end\\
\\
\textbf{Cost of Computing $a_1, a_2, \ldots , a_n$:}
$$ a_{k+1} = \dfrac{y_{k+1} - p_{k}(x_{k+1})}{ (x_{k+1} - x_0)(x_{k+1} - x_1) \ldots (x_{k+1} - x_k) } $$
Cost of computing $a_{k+1}$:
\begin{itemize}
	\item[1] Compute $p_{k}(x_{k+1})$ we need $3k$ flops, so for the numerator needs $3k+1$ flops.
	\item[2] for denominator, we need $k$ flops for multiplication, $k+1$ flops for subtraction.  
\end{itemize}
So, there are total $5k+2 + 1  \text{ ( for division ) } = 5k+3$ flops.\\
\\
Total cost 
$$ \sum_{k = 0}^{n - 1} (5k + 3) = \dfrac{5}{2}n^2 + \dfrac{1}{2} n \approx \dfrac{5}{2} n^2 \ flops $$\\
\\
\textbf{A more Efficient Method for computing $a_0, a_1, a_2, \ldots a_n$ : }\\
As we know that, $p_n(x)$ is always in this form: 
$$ p_n(x) = \sum_{i = 0}^{n} a_i \cdot \prod_{j = 0}^{i - 1} (x - x_j) $$
interpolates $(x_i, y_i) $ for $i = 0, 1, 2, \ldots , n$, we have:
$$p_n(x_i) = y_i, \quad i = 0, 1, 2, \ldots , n$$
Thus, we build the linear system $Aa = y$
$$ 
\begin{bmatrix}
	&1 \\
	&1 & x_1 - x_0 \\
	&1 & x_2 - x_0 & \prod_{j = 0}^1(x_2 - x_j) \\
	&1 & x_3 - x_0 & \prod_{j = 0}^1(x_3 - x_j) & \prod_{j = 0}^2(x_3 - x_j)  \\
	&\vdots &\vdots & \vdots &\vdots\\
	&1 & x_n - x_0 & \prod_{j = 0}^1(x_n - x_j) & \prod_{j = 0}^2(x_n - x_j) &\ldots & \prod_{j = 0}^{n-1}(x_n - x_j)    \\
\end{bmatrix} \cdot \begin{bmatrix}
	a_0\\
	a_1\\
	a_2\\
	a_3\\
	\vdots\\
	a_n
\end{bmatrix}
 = 
 \begin{bmatrix}
	y_0\\
	y_1\\
	y_2\\
	y_3\\
	\vdots\\
	y_n
\end{bmatrix} $$







\newpage

\begin{algorithm}
\caption{Newton's Approach}
\begin{algorithmic}[1]
\State Input: nodes $(x_i, y_i)$\\

\For{ $k = 0 : n - 1$}
	\State $a_k \gets y_k$
	\For{ $ i = k+1 : n$ }
		\State $y_i \gets \dfrac{y_i - y_k}{x_i - x_k} $
	\EndFor

\EndFor




\end{algorithmic}
\end{algorithm}

For the inner "For-Loop", there are $3(n - k)$ flops.\\
Thus there are total:
$$ \sum_{k = 0}^{n - 1} 3\cdot(n - k) = \dfrac{3}{2} n \cdot (n+1) \approx \dfrac{3}{2}n^2 $$



\section*{EXAMPLE:}

\begin{center}
\begin{tabular}{ c|cccc } 
	
 $x$ & $-2$ &$0$ &$1$ &$2$\\ 
 \hline
 $y$ & 2 &4 & 2 &2 \\ 

\end{tabular}
\end{center}

\subsection*{The Vandermonde approach:}
Let $p(x) = c_0 + c_1 \cdot x + c_2 \cdot x^2 + c_3 x^3$\\
\\
$$ \begin{cases}
	p(-2) = 2\\
	p(0) = 4\\
	p(1) = 2\\
	p(2) = 2
\end{cases} \implies \begin{cases}
	c_0 - 2\cdot c_1 + 4\cdot c_2 - 8\cdot c_3 = 2\\
	c_0 + 1\cdot c_1 + 1\cdot c_2 + 1\cdot c_3 = 2\\
	c_0 + 2\cdot c_1 + 4\cdot c_2 + 8\cdot c_3 = 2\\
	c_0 = 4
\end{cases} $$
Thus, we have 
$$c_0 = 4 \quad c_1 = - 2 \quad c_2 = -\dfrac{1}{2} \quad c_3 =  \dfrac{1}{2}$$
Thus, we have $$p(x) =  \dfrac{1}{2}x^3 - \dfrac{1}{2}x^2 - 2x + 4 $$
\subsection*{The Lagrange approach}
We write $p(x)$ in the Lagrange Form
$$p(x) = l_0(x)y_0 + l_1(x)y_1 + l_2(x)y_2 + l_3(x)y_3 \quad  where  $$
\begin{align*}
	l_0(x) &= \dfrac{(x - x_1)(x - x_2)(x - x_3)}{(x_0 - x_1)(x_0 - x_2)(x_0 - x_3)} = \dfrac{(x - 0)(x - 1)(x - 2)}{(-2 - 0)(-2 - 1)(-2 - 2)} = -\dfrac{1}{24} \cdot ( x^3 - 3\cdot x^2 + 2\cdot x )\\ 
	l_1(x) &= \dfrac{(x - x_0)(x - x_2)(x - x_3)}{(x_1 - x_0)(x_1 - x_2)(x_1 - x_3)} = \dfrac{(x + 2)(x - 1)(x - 2)}{(0 + 2)(0 - 1)(0 - 2)} = \dfrac{1}{4} \cdot ( x^3 -  x^2 - 4\cdot x + 4)\\ 
	l_2(x) &= \dfrac{(x - x_0)(x - x_1)(x - x_3)}{(x_2 - x_0)(x_2 - x_1)(x_2 - x_3)} = \dfrac{(x + 2)(x - 0)(x - 2)}{(1 + 2)(1 - 0)(1 - 2)} = -\dfrac{1}{3} \cdot ( x^3 - 4\cdot x )\\ 
	l_3(x) &= \dfrac{(x - x_0)(x - x_1)(x - x_2)}{(x_3 - x_0)(x_3 - x_1)(x_3 - x_2)} = \dfrac{(x + 2)(x - 0)(x - 1)}{(2 + 2)(2 - 0)(2 - 1)} = \dfrac{1}{8} \cdot ( x^3 + x^2 - 2\cdot x )\\ 
\end{align*}
Thus, we have 
\begin{align*}
	p(x) &= 2 \cdot (-\dfrac{1}{24}) \cdot ( x^3 - 3\cdot x^2 + 2\cdot x )\\ 
		 &+ 4 \cdot \dfrac{1}{4} \cdot ( x^3 -  x^2 - 4\cdot x + 4)\\ 
		 &+ 2 \cdot (-\dfrac{1}{3}) \cdot ( x^3 - 4\cdot x )\\ 
		 &+ 2 \cdot \dfrac{1}{8} \cdot ( x^3 + x^2 - 2\cdot x )\\ 
		 &= \dfrac{1}{2} x^3 - \dfrac{1}{2} x^2 -2x + 4
\end{align*}
\subsection*{Newton form}
Suppose that 
$$p(x) = a_0 + a_1(x - x_0) + a_2(x-x_0)(x- x_1) + a_3(x - x_0)(x - x_1)(x - x_2)$$
The coefficient is calculated according to the algorithm given in class:
\begin{itemize}
	\item $x_0 = -2\ \  y_0 = 2 \quad \quad a_0: a_0 = 2$
	\item $x_1 = 0\ \  y_1 = 4 \quad \quad a_1: \dfrac{4 - 2}{0 - (-2)} = 1 \implies a_1 = 1$
	\item $x_2 = 1\ \  y_2 = 2 \quad \quad a_2: \dfrac{2-2}{1-(-2)} = 0 \quad \dfrac{0 - 1}{1 - 0} = -1 \implies a_2 = -1$
	\item $x_3 = 2\ \  y_3 = 2 \quad \quad  a_3: \dfrac{2-2}{2-(-2)} = 0 \quad \dfrac{0 - 1}{2 - 0} = -\dfrac{1}{2} \quad \dfrac{-1/2 - (-1)}{2 - 1} = \dfrac{1}{2} \implies a_3 = \dfrac{1}{2}$
\end{itemize}
So, we can say that $a_0 = 2 \quad a_1 = 1 \quad a_2 = -1 \quad a_3 = \dfrac{1}{2}$\\
\\
Thus, we have:
$$p(x) = 2 + 1\cdot (x + 2) -1 \cdot (x+2)(x- 0) + \dfrac{1}{2}\cdot (x + 2)(x - 0)(x - 1)$$
$$p(x) = \dfrac{1}{2} x^3 - \dfrac{1}{2} x^2 -2x + 4 $$



\textbf{Question: Why a high degree Polynomial Interpolation is not a good idea ? }
\begin{mdframed}
	$f$ will not well approximate at all intermediate points as the number of nodes increases.\\
	The polynomial may be far away from the function at some point.
	\\
	\\
	Take \textit{ Runge function } for example:
	$$ f(x) = \dfrac{1}{1 + 25x^2}, x \in [-1, 1] $$
	If $p_n$ is the polynomial that interpolates the $f$ at $n+1$ equally spaced points on $[-1, 1]$, then
	$$ \lim_{n \to \infty} \max_{-1 \leq x \leq 1} |f(x) - p_n(x)| = \infty $$
\end{mdframed}










\end{document}