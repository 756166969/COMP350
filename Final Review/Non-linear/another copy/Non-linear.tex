\documentclass [9 pt]{article}
\usepackage[margin = 1in]{geometry}
\usepackage{amsfonts}
\usepackage{amsthm}
\usepackage{bbm}
\usepackage{amsmath}
\usepackage{arydshln}
\usepackage[utf8]{inputenc}
\usepackage{graphicx}
\usepackage{enumerate}
\usepackage{color}
\usepackage[dvipsnames]{xcolor}
\usepackage{graphicx}
\graphicspath{ {./images/} }
\usepackage{tikz}
\usepackage{xcolor}
\usepackage{listings}
\usepackage{color}
\usepackage{algorithm}
\usepackage{algpseudocode}
\usepackage{tikz}
\usepackage{booktabs}
\usepackage[framemethod = tikz]{mdframed}

\definecolor{dkgreen}{rgb}{0,0.6,0}
\definecolor{gray}{rgb}{0.5,0.5,0.5}
\definecolor{mauve}{rgb}{0.58,0,0.82}


\lstset{
  language=Matlab,
  basicstyle=\ttfamily,               
  numbers=left,                  
  stepnumber=1,                  
  numbersep=5pt,                  
  backgroundcolor=\color{white}, 
  showspaces=false,              
  showstringspaces=false,        
  showtabs=false,                
  tabsize=2,                     
  captionpos=b,                  
  breaklines=true,               
  breakatwhitespace=true,         
  title=\lstname,
  numberstyle=\tiny\color{Black},
  keywordstyle=\color{BrickRed},
  commentstyle=\color{dkgreen},
  stringstyle=\color{mauve},  
}

\theoremstyle{definition}
\newtheorem{problem}{Problem}
\newtheorem{theorem}{Theorem}
\newtheorem*{corollary}{Corollary}
\newtheorem{proposition}[theorem]{Proposition}
\newtheorem{lemma}[theorem]{Lemma}
\newtheorem{conjecture}[theorem]{Conjecture}

\newtheorem{definition}[theorem]{Definition}
\newtheorem{remark}[theorem]{Remark}
\newtheorem{example}[theorem]{Example}


\usepackage{fancyhdr}
\pagestyle{fancy}
\lhead{Yuhao Wu \quad 260711365} 
\rhead{\bfseries COMP 350 Final Review 3}
\cfoot{\thepage}
\renewcommand{\headrulewidth}{0.4pt}
\renewcommand{\footrulewidth}{0.4pt}


\setlength{\parindent}{0pt}


\begin{document}

\title{Solving Non-linear Equations (20 points)}
\date{2018-11-30}
\author{Name: Yuhao Wu\\
ID Number: 260711365
}
\maketitle

\textbf{Question: Why we need iterative methods?}
\begin{mdframed}
	If $f$ is a general nonlinear function, there is no formula exists for finding roots of $f(x) = 0$, \textbf{iterative methods } will be used to compute \textbf{ approximate roots }
\end{mdframed}


\section*{Three algorithms:}

\subsection*{Bisection algorithm:}

\begin{algorithm}
\caption{Bisection algorithm}
\begin{algorithmic}[1]
\State Input: $f(x)\quad [a, b]$ with $f(a) \cdot f(b)<0$ \quad the tolerance $\delta$

	\State $c \gets \dfrac{a + b}{2}$ \Comment{c is the root we will return}
	\State $error\ bound  \gets  \dfrac{|b - a|}{2} $ 
	
	\While{ $ error\ bound > \delta$ }
		\If {$ f(c) = 0 $}
			\State we say $c$ is the root, stop.
			\Else
				\If{$ f(a) \cdot f(c) < 0 $}
					\State $b \gets c$ 
					\Else
					\State $a \gets c$
				\EndIf	
		\EndIf
		\State $c \gets \dfrac{a + b}{2}$
		\State $error\ bound \gets \dfrac{error\ bound}{2}$
		
	\EndWhile
	
	\State \Return c


\end{algorithmic}
\end{algorithm}


\newpage

\subsection*{Newton's Method:}
\begin{algorithm}
\caption{Newton's algorithm}

\begin{algorithmic}[1]
\State Input: $f(x) \quad f'(x), x, xtol, ftol, n_{max} $ \Comment{$x \to$ initial point }
\State \Comment{$ n_{max} \to$ maximum number of iterations }

	\For{$n = 1 : n_{max} $}
	\\
		\State $d \gets \dfrac{f(x)}{ f'(x)} $ 
		\\
		\State $x \gets x - d$
		
		\If{ $ |d| \leq xtol \ or \ |f(x)| \leq ftol $ }
			\State $root \gets x$
			\State stop
		\EndIf
	\\
	\EndFor	
	
	\State $root \gets x$


\end{algorithmic}
\end{algorithm}


Stop Criteria:
\begin{itemize}
	\item $|x_{n+1} - x_n| \leq xtol$
	\item $|f_{n+1} | \leq ftol$
	\item The maximum number of iteration reached
\end{itemize}


\textbf{Newton's Method's Iteration formula:}
\begin{mdframed}
	Given a point $x_0$, we do Taylor Expansion about $x_0$:
	$$f(x) = f(x_0) + f'(x_0)\cdot ( x - x_0) + \dfrac{f''(z)}{2} (x - x_0)^2 $$
	where $z \in (x, x_0)$\\
	\\
	We use $\ell(x) = f(x_0) + f'(x_0)\cdot ( x - x_0)$ as an approximation of $f(x)$. In order to solve $f(x) = 0$, we try to solve $\ell(x) = 0$
	$$\ell(x) = 0 \implies  f(x_0) + f'(x_0)\cdot ( x - x_0) = 0 \implies 
	x = x_0 - \dfrac{f(x_0)}{f'(x_0)}  $$
	We choose $x_1 = x_0 - \dfrac{f(x_0)}{f'(x_0)}   $ as our approximate root of $f(x) = 0$\\
	If this $x_1$ is not a good approximate root, we continue this iteration. In general, we have the Newton iteration:
	$$x_{n+1} = x_n - \dfrac{f(x_n)}{f'(x_n)}$$
\end{mdframed}






\newpage
\subsection*{Secant Method:}
Sometimes, the derivative of a function is really hard or even impossible to compute, then, we try to use \textbf{ divided difference } to replace $f'(x_n)$. Then we get our secant iteration:
$$ x_{n+1} = x_n - \dfrac{f(x_n)}{f'(x_n)} \quad\text{ use \textbf{ divided difference } to replace $f'(x_n)$} \implies \boxed{  x_{n+1} = x_n - \dfrac{x_n - x_{n - 1}}{f(x_n) - f(x_{n - 1})} \cdot f(x_n) } $$
\begin{algorithm}
\caption{Secant's algorithm}

\begin{algorithmic}[1]
\State Input: $f(x) , x_0, x_1, xtol, ftol, n_{max} $ \Comment{$x_0, x_1 \to$ initial point }
\State \Comment{$ n_{max} \to$ maximum number of iterations }

	\For{$n = 1 : n_{max} $}
	\\
		\State $d \gets \dfrac{x_1 - x_0}{f(x_1) - f(x_0)} \cdot f(x_1)  $ 
		\\
		\State $x_0 \gets x_1 $
		\State $f(x_0) \gets f(x_1)$
		\State $x_1 \gets x_1 - d$
		\State $fx_1 = f(x_1) $
		\\
		\If{ $ |d| \leq xtol \ or \ |fx_1| \leq ftol $ }
			\State $root \gets x_1$
			\State stop
		\EndIf
	\\
	\EndFor	
	
	\State $root \gets x_1$


\end{algorithmic}
\end{algorithm}

\section*{Comparison of the Three Methods:}
\begin{itemize}
	\item The Bisection Method:
		\begin{itemize}
			\item Advantage: Simple, guaranteed to converge, applicable to non-smooth functions
			\item Disadvantage: Linear convergence
		\end{itemize}
	\item Newton's Method:
		\begin{itemize}
			\item Advantage: Quadratic convergence
			\item Disadvantage: need to compute $f'$ \quad may not converge
		\end{itemize}
	\item The Secant Method:
		\begin{itemize}
			\item Advantage: Super-Linear convergence\quad no need to compute $f'$
			\item Disadvantage: slower than $NM$ \quad may not converge
		\end{itemize}
\end{itemize}


\newpage
\section*{Convergence Analysis:}
\textbf{Quadratic Convergence:}
A sequence $\{ x_n \}$ is said o have quadratic convergence to $x$ if:
$$ \lim_{n \to \infty} \dfrac{|x_{n+1} - x|}{ |x_n - x|^2 } = c $$
where $c$ is some finite non-zero constant.\\
\\
Suppose $r$ is a root of $f(x) = 0$ and $x_n$ converges to $r$.\\
We do Taylor series expansion about $x_n$
$$f(r) = f(x_n) + (r - x_n) f'(x_n) + \dfrac{(r - x_n)^2}{2} f''(z_n)$$
where $z_n$ is between $r$ and $x_n$.\\
\\
Divide both sides by $f'(x_n)$ and use the fact $r$ is the root of $f(x) = 0$, then we have:
$$ 0 = \dfrac{f(x_n)}{f'(x_n)} + (r - x_n) + (r - x_n)^2 \cdot \dfrac{f''(z_n)}{2f'(x_n)} $$
As we use Newton's Method, we have $x_{n+1} = x_n - \dfrac{f(x_n)}{f'(x_n)} $, then we have:
$$ 0 = x_n - x_{n+1} + (r - x_n) + (r - x_n)^2 \cdot \dfrac{f''(z_n)}{2f'(x_n)} \implies r - x_{n+1} = c_n \cdot (r - x_n)^2, \quad c_n = - \dfrac{f''(z_n)}{2f'(x_n)} $$
As we said before, $x_n \to r$ and $z_n $ is between $x_n$ and $r$, thus we have $z_n \to r$ and then, we have:
$$ \lim_{n \to \infty} \dfrac{|r - x_{n+1}|}{ |r - x_n|^2 } = \lim_{n \to \infty} |c_n| = |\dfrac{f''(r)}{2f'(r)}|  = c $$

\subsection*{Show that $\lim_{n\to \infty} | \dfrac{f(x_{n+1})}{f(x_n)} | = c$}
\begin{mdframed}
	We expand the Taylor Series as following:
$$ f(x_{n+1}) = f(x_n) + (x_{n+1} - x_n )\cdot f'(x_n) + \dfrac{(x_{n+1} - x_{n})^2}{2}\cdot f''(z_n)\ where\  z_n \text{ is between } x_n \ and\ x_{n+1} $$
According to Newton Iteration, we have:
$$ x_{n+1} = x_n - \dfrac{f(x_n)}{f'(x_n)}  \implies x_n - x_{n+1} =   \dfrac{f(x_n)}{f'(x_n)}  $$
Thus, we have:
$$ f(x_{n+1}) = f(x_n) - f(x_n) + \dfrac{ 1 }{2} \cdot \dfrac{f(x_n)^2}{f'(x_n)^2} \cdot  f''(z_n)  $$
$$ \dfrac{f(x_{n+1})}{f(x_n)^2} = \dfrac{1}{2} \cdot \dfrac{f''(z_n)}{f'(x_n)^2} $$
As, we know, $  z_n \text{ is between } x_n \ and\ x_{n+1} $, when $n \to \infty$, we have $x_n$ and $x_{n+1}$ converges to root $r$, which means, $z_n \to r$\\
\\
Thus, we have:
$$ \lim_{n \to \infty}  \dfrac{|f(x_{n+1})|}{|f(x_n)^2|} = \lim_{n \to \infty} \dfrac{1}{2} \cdot \dfrac{|f''(z_n)|}{|f'(x_n)|^2} = \dfrac{1}{2} \cdot \dfrac{|f''(r)|}{|f'(r)|^2} = c \neq 0 $$
\end{mdframed}




\bigskip
\section*{Some Questions about this Chapter:}
\begin{itemize}
	\item[1] Under which cases, NM and SM are not converge?\\
	Answer: the initial point is not close to the root $r$
	\item[2] Do we need to know how to derive the convergence rate of Bisection is linear?
	(  No. )
\end{itemize}

\bigskip

 Usually? What about unusual case?
	$$c = |\dfrac{f''(r)}{2f'(r)}| $$
	So, could I say unusual case always comes from one of $f''(r) \quad f'(r)$ = 0\\
	\\
	$f''(r) = 0 \implies$ converge faster than quadratic\\
	$f'(r) = 0 \implies $ converge at linear convergence rate, but why?\\
	\\
	

\begin{mdframed}
	For $f(x) = (x - 1)^2$, it converges at linear rate.\\
	What we want to prove is that
	\begin{equation}
		\lim_{n \to \infty} \dfrac{|x_{n+1} - x|}{ |x_n - x|} = \text{some constant } c 
	\end{equation}
	As we used in Newton method:
	$$ x_{n+1} = x_n - \dfrac{f(x_n)}{f'(x_n)} = x_n - \dfrac{(x_n - 1)^2}{2\cdot (x_n - 1)}  = x_n - \dfrac{x_n - 1}{2} = \dfrac{x_n + 1}{2}$$
	Thus, we substitute this into Equation(1):
	$$ \lim_{n \to \infty} |\dfrac{\dfrac{x_n + 1}{2} - 1}{x_n - 1}| = \dfrac{1}{2} $$
\end{mdframed}

\newpage
\begin{mdframed}
	For $f(x) = \sin(x)$, it converges faster than Quadratic Rate.\\
		What we want to prove is that
	\begin{equation}
		\lim_{n \to \infty} \dfrac{|x_{n+1} - x|}{ |x_n - x|^ k} = \text{some constant } c 
	\end{equation}
	As we used in Newton method:
	$$ x_{n+1} = x_n - \dfrac{f(x_n)}{f'(x_n)} = x_n - \dfrac{\cos(x_n)}{\sin(x_n)} $$
	Thus, we substitute this into Equation(2):
	$$ \lim_{n \to \infty}\dfrac{|x_{n+1} - x|}{ |x_n - x|^ k} = \dfrac{|x_n - \dfrac{\cos(x_n)}{\sin(x_n)} - \pi|}{|x_n - \pi|^k}  $$
\end{mdframed}









\end{document}
