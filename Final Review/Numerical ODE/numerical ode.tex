\documentclass [9 pt]{article}
\usepackage[margin = 1in]{geometry}
\usepackage{amsfonts}
\usepackage{amsthm}
\usepackage{bbm}
\usepackage{amsmath}
\usepackage{arydshln}
\usepackage[utf8]{inputenc}
\usepackage{graphicx}
\usepackage{enumerate}
\usepackage{color}
\usepackage[dvipsnames]{xcolor}
\usepackage{graphicx}
\graphicspath{ {./images/} }
\usepackage{tikz}
\usepackage{xcolor}
\usepackage{listings}
\usepackage{color}
\usepackage{algorithm}
\usepackage{algpseudocode}
\usepackage{tikz}
\usepackage{booktabs}
\usepackage[framemethod = tikz]{mdframed}

\definecolor{dkgreen}{rgb}{0,0.6,0}
\definecolor{gray}{rgb}{0.5,0.5,0.5}
\definecolor{mauve}{rgb}{0.58,0,0.82}


\lstset{
  language=Matlab,
  basicstyle=\ttfamily,               
  numbers=left,                  
  stepnumber=1,                  
  numbersep=5pt,                  
  backgroundcolor=\color{white}, 
  showspaces=false,              
  showstringspaces=false,        
  showtabs=false,                
  tabsize=2,                     
  captionpos=b,                  
  breaklines=true,               
  breakatwhitespace=true,         
  title=\lstname,
  numberstyle=\tiny\color{Black},
  keywordstyle=\color{BrickRed},
  commentstyle=\color{dkgreen},
  stringstyle=\color{mauve},  
}

\theoremstyle{definition}
\newtheorem{problem}{Problem}
\newtheorem{theorem}{Theorem}
\newtheorem*{corollary}{Corollary}
\newtheorem{proposition}[theorem]{Proposition}
\newtheorem{lemma}[theorem]{Lemma}
\newtheorem{conjecture}[theorem]{Conjecture}

\newtheorem{definition}[theorem]{Definition}
\newtheorem{remark}[theorem]{Remark}
\newtheorem{example}[theorem]{Example}


\usepackage{fancyhdr}
\pagestyle{fancy}
\lhead{Yuhao Wu \quad 260711365} 
\rhead{\bfseries COMP 350 Final Review 6}
\cfoot{\thepage}
\renewcommand{\headrulewidth}{0.4pt}
\renewcommand{\footrulewidth}{0.4pt}


\setlength{\parindent}{0pt}


\begin{document}

\title{Numerical ODE (10 points)}
\date{2018-11-30}
\author{Name: Yuhao Wu\\
ID Number: 260711365
}
\maketitle



In this course, we mainly focus on the following general \textit{Initial-Value-Problem} for a first order ODE.

$$ \begin{cases}
	x' = f(x, t)\\
	x(a) = x_0
\end{cases} \quad \quad or \quad \quad 
\begin{cases}
	\dfrac{dx(t)}{dt} = f\Big(x(t), t\Big)\\
	\\
	x(a) = x_0
\end{cases}
$$
In many applications, the closed-form solution for the above IVP may be very complicated and difficult to evaluate or there is no closed-form solution. So, we want a numerical solution.\\
\\
A numerical algorithm for solving an $ODE$ produces a sequence of points $(t_i, x_i), \quad i = 0, 1, 2, \ldots$ where $x_i$ is an approximation to the true value $x(t_i)$, while the mathematical solution is a continuous function $x(t)$\\
\\
\textbf{NOTE:} But you can do polynomial interpolation on the points we have already calculated to get a continuous form.


\section*{Euler's Method:}
We would like to find approximate values of the solution to the IVP over the interval $[a, b]$. Use $n + 1$ points $t_0, t_1, t_2, \ldots , t_n$ to equally partition $[a, b]$. We call $h = t_{i+1} - t_i = \dfrac{b - a}{n}$ \textbf{Size Step}.\\
\\
Suppose that we have already obtained $x_i$, an approximation to $x(t_i)$. We would like to get $x_{i+1}$, an approximation to $x(t_{i+1})$. We use the Taylor Expansion:
$$x(t_{i + 1}) \approx x(t_i) + (t_{i + 1} - t_i)x'(t_i) = x(t_i) + h \cdot f\Big( x(t_i), t_i \Big) $$
leads to the formula of Euler's Method:
$$ x_{i+1} = x_i + h \cdot f(t_i, x_i) $$


\newpage
\begin{algorithm}
\caption{Euler's Method: }
\begin{algorithmic}[1]
\State Input: $f(x, t), [a, b], x_0, n$ \Comment{$x_0$ is value of initial point}
\\
\State $h \gets \dfrac{b - a}{n}$
\State $t_0 \gets a$
\\
\For{$i = 0:n-1$}
	\State $ x_{i+1} \gets x_i + h\cdot f(t_i, x_i) $
	\State $t_{i+1} = t_i + h$
\EndFor

\end{algorithmic}
\end{algorithm}

\subsection*{Error Analysis for Euler's Method:}
By Taylor Expansion, we have:
\begin{equation}
	x(t_{i+1}) = x(t_{i}) + h \cdot f(t_i, x(t_i)) + \dfrac{1}{2} h^2 x''(z_{i + 1}) \quad\quad z_{i+1} \in [t_i, t_{i+1}]
\end{equation}
Euler's Methods gives us:
\begin{equation}
	x_{i+1} = x_i + h\cdot f(t_i, x_i)
\end{equation}
We use $(1) - (2)$, then we have:
$$ x(t_{i+1}) - x_{i+1} = \Big[  x(t_{i}) - x_i \Big] 
		+ h \cdot \Big[ f(t_i, x(t_i)) - f(t_i, x_i) \Big] + \dfrac{1}{2} \cdot h^2 x''(z_{i+1}) \quad \quad z_{i+1} \in [t_i, t_{i+1}] $$

$x(t_{i+1}) - x_{i+1}$ is the error at $t_{i+1}$. This is called the \textbf{ global error } at $t_{i+1}$. It arises from two sources:
\begin{itemize}
	\item The \textbf{Local Truncation Error}: $ \dfrac{1}{2} \cdot h^2 x''(z_{i+1})$ \quad\quad $O(h^2)$
	\item The \textbf{Propagation Error:} $\Big[  x(t_{i}) - x_i \Big] 
		+ h \cdot \Big[ f(t_i, x(t_i)) - f(t_i, x_i) \Big]$. This is due to the accumulated effects of all local truncation errors at $t_1, t_2, \ldots t_i$   
\end{itemize}
When we do computing on computers, there will be \textbf{ rounding errors } as well.


\section*{Trapezoidal Euler's Method:}
For $x'(t) = f\Big(t, x(t)\Big)$, we integral both sides:

\begin{equation}
	 \int_{t_i}^{t_{i+1}} x'(t)\ dt = \int_{t_i}^{t_{i+1}} f\Big(t, x(t)\Big)\ dt \implies x(t_{i+1}) - x_{t_i} = \int_{t_i}^{t_{i+1}} f\Big(t, x(t)\Big)\ dt
\end{equation}

We apply Trapezoid Rule to the right-hand-side, then we have:
$$ x(t_{i+1}) - x_{t_i} \approx \dfrac{1}{2} \cdot h \cdot \Bigg[ f(t_i, x(t_i) + f\Big(t_{i+1}, x( t_{i+1} )\Big) \Bigg] $$
This leads to scheme:
$$ x_{i+1} =  x_{i} + \dfrac{1}{2} \cdot h \cdot \Bigg[ f(t_i, x_i) + f\Big(t_{i+1}, x_{i+1} \Big) \Bigg] $$

\newpage
But, there is another question, the Right-Hand-Side involves $x_{i+1}$, but $x_{i+1}$ is what we want to calculate, to solve this problem, we use \textbf{Euler's Method} to compute $x_{i+1}$ on the RHS, leading to the formula of the \textbf{Trapezoidal Euler's Method}

$$
\begin{cases}
	\widehat{x}_{i+1} = x_i + h\cdot f(t_i, x_i) \\
	\\
	x_{i+1} =  x_{i} + \dfrac{1}{2} \cdot h \cdot \Bigg[ f(t_i, x_i) + f\Big(t_{i+1}, \widehat{x}_{i+1} \Big) \Bigg]
\end{cases}
$$ 
\\
\textbf{The Local Truncation Error} is $O(h^3)$

\section*{Midpoint Euler's Method:}
It is really similar to the \textbf{Trapezoidal Euler's Method}.\\
\\
According to Equation(3), we have:

\begin{equation}
	x(t_{i+1}) - x_{t_i} = \int_{t_i}^{t_{i+1}} f\Big(t, x(t)\Big)\ dt	
\end{equation}
Then, we apply \textbf{ Midpoint Rule } on the RHS, then we have:
$$ 
x(t_{i+1}) - x_{t_i} \approx  h\cdot f(t_i + \dfrac{1}{2}\cdot h, x_{i + 1/2})
$$

This leads to scheme:
$$ x_{i+1} =  x_{i} + h\cdot f\Big(t_i + \dfrac{1}{2}\cdot h,\ x_{i + 1/2}\Big) $$
Then, we use \textbf{Midpoint Rule} to get $x_{1 + 1/2}$, which gives us the formula of the \textbf{Midpoint Euler's Method}:
$$
\begin{cases}
	x_{i + 1/2 } = x_i + \dfrac{1}{2} h \cdot  f\Big(t_i ,\ x_{i }\Big) \\
	\\
	x_{i+1} =  x_{i} + h\cdot f\Big(t_i + \dfrac{1}{2}\cdot h,\ x_{i + 1/2}\Big)
\end{cases}
$$
\\
\textbf{The Local Truncation Error} is $O(h^3)$




















\end{document}